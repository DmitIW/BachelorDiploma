% В этом файле следует писать текст работы, разбивая его на
% разделы (section), подразделы (subsection) и, если нужно,
% главы (chapter).

% Предварительно следует указать необходимую информацию
% в файле SETUP.tex

\input{preamble.tex}

\begin{document}

\Intro

В рамках данной работы освещается вопрос семантического анализа изображений путем применения глубоких сверточных нейронных сетей. Под семантическим анализом понимается получение из изображения какой-либо интерпретируемой информации: расположение объектов на сцене, принадлежность объектов к заранее заданным классам, наличие на изображении объектов определенного класса и т.п.. Данный вопрос будет рассмотрен на примере задачи обнаружения и выделения неба на изображениях. Входными данными задачи являются фотографии, сделанные на камеры мобильных устройств. Специализированный домен изображений был выбран с целью упрощения адаптация потенциального решения к применению в конечных продуктах, таких как клиентское программное обеспечение для смартфонов. Решением задачи выступают сгенерированные для входных фотографий полутоновые изображения. Такое изображение называется сегментационной маской и для каждого пикселя изначльной фотографии выражает его принадлежность к региону неба: белый цвет интерпретируется как положительный результат, черный - как отрицательный. В ходе разарботки алгоритма решения задачи была исселдована эффективность применения различных подходов к генерации подобного рода масок - сегментации. Сравнения эффективности проходило по индексу Жаккара - в западной литературе также встречается название intersection over union, IoU. Для решения задачи сегментации из наиболее эффективных подходов был составлен стек алгоритмов: применение к входному изображению глубокой сверточной сети с последующей корректировкой методами компьютерного зрения полученной маски. В ходе обучения модели искусственной нейронной сети, ИНС, для предотвращения переобучения и улучшения сходимости применялись техники регуляризация, такие как learning rate reduce и one cycle policy. Влияние данных подходов на решение также отражено в результатах работы. В заключительной части хода разработки был рассмотрена возможность адаптация модели, обученной на данных датасета SkyFinder, к выбранному домену изображений. Такого рода возможность упрощает этап подготовки данных.

% Если typeOfWork в SETUP.tex задан как 2 или 3, то начинать
% надо не с section (раздел), а с главы (chapter)
\section{Постановка и описание задачи}
\label{task_description_start}

Обработка и анализ цифровых изображений является комплексной темой. В нее входят задачи фильтрации, цветовых и яркостных преобразований, морфологическая обработка, распознавание и выделение объектов на сцене~\autocite{gonzalez2008digital}. Семантический анализу - подход, целью которого является получение интерпретируемой информации высших порядков об изображении. К подзадачам семантического анализа можно отнести классификацию объектов на сцене, детекцию объектов и сегментацию изображений на семантические регионы. В рамках текущей работы будет проведен разбор подзадачи сегментации.

\subsection{Задача сегментации}

Для возможности цифровой обработки и анализа будем рассматривать представление изображения как трехмерного массива чисел, имеющего ширину, количество столбцов, и высоту, количество строк, равными ширине и высоте изображения соответсвенно. На каждой позиции по ширине и высоте будет находится вектор из трех целых чисел, из промежутка [0, 255], что соответсвует RGB модели представления цвета пикселя изображения. В таком случае сегментацией изображения будет являться отображения каждого такого RGB вектора в некоторое целое число. Это число может соответствовать идентификатору некоторого класса. При применение такого отображения ко всему изображению получается двумерный массив, ширина и высота которого соответсвуют таковым у исходного изображения. Подобный двумерный массив будет разделять изображение на регионы по обозначенному признаку и иметь нзавание маски сегментации. Сегментацию возможно рассматривать как попиксельную классификацию объектов.

Отметим, что описанное выше отображение может обладать относительно простой природой и учитывать только значение пикселя в конкретной позиции, так и более сложной структурой, использующей информацию о распределении цветов во всем изображении, положении пикселя на изображении и свойствах соседних пикселей, непосредственно соседствующих с обозреваемым значением или отсупающих от него на заданное смещение~\autocite{liu2018recent}.

\subsection{Выделение неба в рамках задачи сегментации}

Описанная в введение задача выделения региона неба на входном изображении может быть рассмотрена как задача сегментации. Так как в данном случае результирующих классов всегда два - класс принадлежности и обратный ему -, то имеется более узкий случай сегментации - бинарная. Итоговая маска будет содержать только значения 0 и 1, представляя собой однобитовое бинарное изображение, что можно считать вырожденным полутоновым.

Таким образом, решение задачи выделения неба сводится к нахождению отображения из вектора цветов для каждого пикселя в целое число из промежутка [0, 1]. Данное отображение возможно получить как алгоритмами машинного зрения, так и с помощью использования моделей глубоких сверточных нейронных сетей. В данной работе приводится решение методом нейронных сетей, при этом алгоритмы компьютерного зрения используются для корректировки полученной маски. Под корректировкой здесь понимается обнаружение и удаление ложноположительных регионов неба и уточнение границы между регионами разных классов.

Практическое применение подобного решения можно найти в пользовательских приложениях эстетической обработки пейзажных фотографий для смартфонов, в автоматических системах мониторинга воздушного пространства, при извлечении семантической информации высшего порядка для использования в иных методах анализа и обработки изображений~\autocite{7415405}.

\section{Обзор существующих исследований}

Задача семантического анализа изображений имеет широкое применение при решении различных прикладных и исследовательских проблем~\autocite{maier2018gentle}~\autocite{pan2019image}~\autocite{stabinger2020evaluating}~\autocite{li2015brief}, в связи с чем активно изучается. Сегментации изображений, в частности, нашли применения в таких областях как проведение хирургических операций, автопилотируемый транспорт, автоматизированное картографирование местности~\autocite{liu2018recent}. Ниже приведен краткий обзор исследований, темы которых связаны с поставленной задачей. Результаты этих исследований в разной степени использовались для построения решения. 

\subsection{Сегментация изображений}

До начала активного применения глубоких нейронных сетей в задачах сегментации использовались методы компьютерного зрения, основанные на применении порогов бинаризации для полутоновых изображений, выявлении признаков, кластеризации методом k-средних~\autocite{10.5555/1888028.1888043}~\autocite{10.5555/540298}~\autocite{inproceedings}. Каждый из этих подходов имеет свои приемущества, метод бинарной сегментации полутоновых изображений до сих пор успешно применяется в области анализа медицинских данных~\autocite{bookMedicalImages}. Но применение данных подходов к задаче выделения границы между объектами путем сегментации показало худшие результаты в сравнении с FCN, полностью сверточными глубокими сетями~\autocite{7966418}.

Современные решения задачи сегментации в различных областях зачастую опираются на применение нейронных сетей~\autocite{feng2019deep}. Наиболее распространенными архитектурами выступают Unet, DeepLab, RefineNet~\autocite{ronneberger2015unet}~\autocite{chen2016deeplab}~\autocite{lin2016refinenet}. Имеются исследования применения архитектуры RefineNet для определния региона неба на датасете SkyFinder~\autocite{place2017segmenting}.


\subsection{Методы регуляризации}

При достаточной сложности модели, в процессе обучения она может начать отражать в ответах шум в тренировачных данных~\autocite{salman2019overfitting}~\autocite{ghojogh2019theory}. Данное явление называется переобучением, overfitting. С целью снизить вероятность его проявление применяются разнообразные техники регуляризации: dropout, нормализация значений между слоями сети, настройка гиперпараметров процесса обучения~\autocite{smith2018disciplined}~\autocite{labach2019survey}~\autocite{ioffe2015batch}.

\section{Данные}

Эксперименты с обучением модели ИНС проводились на двух наборах данных: SkyFinder и набор из фотографий, сделанных на камеры смартфонов, с синтетической разметкой в объединении с данными из SkyFinder. 

\subsection{Датасет SkyFinder}

SkyFinder представляет собой набор из 90.000 фотографий~\autocite{mihail2016sky}. Все фотографии сделаны на статичные веб-камеры, расположенные вне зданий. В верхней части изображений преобладает регион неба. Средний процент пикселей относящихся к классу принадлжености равен 41 со стандартным отклонением в 16 процентов. Изображения покрывают широкий диапозон освещенности и погодных условий, что препятсвует переобучению на конкретных значениях. Для каждого изображения имеется размеченная бинарная маска с описанным в постановке задачи свойством: 1 обозначает класс принадлежности, 0 - обратный ему. 

\subsection{Датасет с синтетической разметкой}



\Conc

% Печать списка литературы (библиографии)
\printbibliography[%{}
    heading=bibintoc%
    %,title=Библиография % если хочется это слово
]
% Файл со списком литературы: biblio.bib
% Подробно по оформлению библиографии:
% см. документацию к пакету biblatex-gost
% http://ctan.mirrorcatalogs.com/macros/latex/exptl/biblatex-contrib/biblatex-gost/doc/biblatex-gost.pdf
% и огромное количество примеров там же:
% http://mirror.macomnet.net/pub/CTAN/macros/latex/contrib/biblatex-contrib/biblatex-gost/doc/biblatex-gost-examples.pdf

\end{document}
